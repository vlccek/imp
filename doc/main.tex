 
\documentclass{article}

\usepackage[czech]{babel}
\usepackage{amsmath}
\usepackage[a4paper,top=2cm,bottom=2cm,left=3cm,right=3cm,marginparwidth=1.75cm]{geometry}
\usepackage[utf8]{inputenc}
\usepackage[czech]{babel}
\usepackage{datetime}
\usepackage{hyperref}

\title{Projekt do předmětu IMP - Š - ESP32 či jiný HW: Digitální FM rádio \\ (modul RDA5807M a zesilovač PAM8407)}

\author{Jakub Vlk}
\date{\today}

\begin{document}
	
	\maketitle
	\newpage
	
	\section{Úvod}
	Tato dokumentace slouží k popisu výsledné implementace projektu zaměřeného na realizaci digitálního FM rádia s využitím vývojového kitu na bázi SoC ESP32, modulu FM přijímače RDA5807M a modulu audio zesilovače PAM8407.
	
	\subsection{Popis projektu}
	
	Na displeji je možné pomocí rotačního enkodéru procházet mezi %TODO
	
	Cílem tohoto projektu je vytvořit funkční digitální FM rádio, které umožní uživatelům naladit a poslouchat rádiové stanice prostřednictvím minimalistického rozhraní na displeji. Další možností ovládání je ovládání skrze webové rozhraní. Uživatelské rozhraní bude možné procházet pomocí rotačního enkodéru.
	
	\subsection{Zapojení}
	Displej je připojen k SoC pomocí rozhraní SPI. Samotný rádiový přijímač RDA5807M je připojen pomocí rozhraní I2C ke SoC pro ladění a nastavování dalších parametrů rádiového přijímače. Rotační enkodér je připojen ke zbývajícím volným GPIO pinům, které byly přenastaveny tak, aby odpovídaly požadavkům enkodéru. Audio zesilovač byl zapojen přímo k FM přijímači a napájení. Samotný audio zesilovač není mimo napájení připojen k SoC. Tyto piny jsou konfigurovatelné v souboru \texttt{settings.hpp}.
	
	
	\section{Rozhraní}
	\subsection{Webové rozhraní}
	Z webového rozhraní je možné pohodlně ovládat rádio na velkou vzdálenost. Stačí, aby systém byl připojen k wi-fi síti. Potom je na adrese systému na portu 80 možno načíst webovou stránku s ovládacími prvky. Ve webovém rozhraní je možné měnit zvolenou stanici, nastavovat hlasitost a zapnout a vypnout funkci "bassboost".
	
	\subsection{Displej}
	Na displeji je možné procházet mezi 3 záložkami -- \textbf{Radio}, \textbf{Favorite} a \textbf{Volume}. V záložce Radio je možné ručně naladit konkrétní frekvenci. Záložka Favorite je určena k procházení mezi 10 přednastavenými oblíbenými stanicemi. V poslední záložce Volume je možné nastavovat hlasitost. Vybraná záložka je indikována velkými písmeny ve slově, které tuto záložku reprezentuje. 
	
	\section{Implementace}
	\subsection{Zapínání radia}
	Po zapnutí systému proběhne inicializace všech komponent. V této fázi jsou nastaveny porty GPIO tak, aby odpovídaly zapojení komponent. Rovněž jsou vytvořeny objekty pro práci s danými komponentami, jako je displej nebo rádio, a také abstraktní komponenty, jako je například menu. Dále se systém připojí k wi-fi síti (parametry jsou definovány v souboru \texttt{settings.hpp}) a následně je inicializován webový server. 
	
	\subsection{Hlavní smyčka programu}
	V těle programu se nachází jen dvě neelementární operace: vykreslení menu a funkce realizující obsluhu webového serveru. Vykreslení menu je více popsáno v sekci \ref{menu}. O implementaci webového rozhraní se píše v sekci \ref{web}.
	
	
	\subsection{Enkodér}
	Ovládaní enkodéru je realizováno bez použití knihovny pomocí obsluhy přerušení. Enkodér může vygenerovat dva druhy přerušení. Prvním je stisknutí tlačítka, což systém interpretuje jako potvrzení nebo vracení zpět a následně detekci otáčení. Směr otáčení se určí podle předchozí hodnoty v jednom z registrů. Po detekci otáčení je zavolána jedna ze dvou metod, a to buď pro točení doprava nebo doleva. Třída \emph{menu} toto otáčení interpretuje -- buď jako pohyb mezi záložkami v menu nebo jako inkrementaci či dekrementaci hodnot v jednotlivých záložkách. Taktéž je nastaven příznak \verb|update|, aby se menu vykreslilo pozměněné. 
	
	\subsection{Menu}
	\label{menu}
	Menu vytváří abstrakci nad směsí textu, čísel a čar zobrazovaných v displeji. Menu si udržuje odkaz jak na displej, tak na rádio a překládá pokyny zadané uživatelem -- jak z enkodérum tak z webového rozhraní -- na displej, ale zároveň i na rádio. Menu pracuje, tak že vygenerované přerušení změní vnitřní stav objektu třídy \emph{menu} a nastaví příznak potřeby překreslení. Samotné přerušení neprovede znovuvykreslení -- to je realizováno až v hlavní smyčce programu při zavolání funkce pro překreslení, aby obsluha přerušení zabrala co možná nejméně času. 
	
	Vykreslení funguje tak, že pouze pokud je nastaven příznak informující, že je potřeba překreslit \verb|update|, tak pouze tehdy proběhne překreslení. Jinak není prováděno nic. Pokud je překreslení potřeba, proběhne nejdříve vykreslení prvního řádku - položek menu. Potom se podle vybrané položky v menu vybere, který z detailů se má vykreslit. Displej je rovněž aktualizován, pokud proběhne změna na webovém rozhraní. 
	
	\subsection{Webové rozhraní} 
	\label{web}
	Webové rozhraní je realizováno jako jednoduchá webová stránka. Pomocí tlačítek a textových polí je možno upravovat nastavení rádia. Pro implementaci byla použita knihovna. Každý požadavek je předán třídě  \emph{menu} (viz \ref{menu}) pro aplikování nastavení a aktualizaci displeje. 
	
	\section{Implementační detaily}
	Za účelem jednoduššího vývoje byla napsána jednoduchá protokolovací makra \verb|FATAL_ERROR|, \verb|ERROR| a \verb|loging|, která vypisují mimo zadané hlášky i řádek a funkci, ve které bylo toto makro zavoláno. Makro \verb|FATAL_ERROR| mimo to provede po vypsané hlášky i restart SOC. Při vývoji se zjistilo, že způsob protokolování je časově náročný, proto je možné jej vypnout. 
	
	Protokolování lze vypnout v souboru \texttt{settings.hpp}, kde se kromě možností pro vypnutí protokolování nachází také všechny konstanty používané v projektu pro jednoduché úpravy počátečního stavu radia.
	
	\subsection{Použité knihovny}
	\begin{enumerate}
		\item Adafruit\_SSD1306 - použita pro ovládaní displeje. Dostupná na \href{https://github.com/adafruit/Adafruit_SSD1306}{githubu}
		\item Radio Library - použita pro ovládaní radia. Dostupná na \href{https://github.com/mathertel/Radio/tree/master}{githubu}
	\end{enumerate}
	
	\section{ Demonstrační video }
	k nalezení na youtube \href{https://youtu.be/VmVauFg2x14}{zde}
	
	
	
\end{document}
